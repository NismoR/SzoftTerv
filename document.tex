\documentclass[11pt,a4paper,oneside]{report}             % Single-side
%\documentclass[11pt,a4paper,twoside,openright]{report}  % Duplex

\PassOptionsToPackage{chapternumber=Huordinal}{magyar.ldf}
\usepackage{t1enc}
%\usepackage[latin2]{inputenc}
\usepackage[utf8]{inputenc}
\usepackage{amsmath}
\usepackage{amssymb}
\usepackage{enumerate}
\usepackage[thmmarks]{ntheorem}
\usepackage{graphics}
\usepackage{epsfig}
\usepackage{listings}
\usepackage{color}
%\usepackage{fancyhdr}
\usepackage{lastpage}
\usepackage{anysize}
\usepackage[magyar]{babel}
\usepackage{sectsty}
\usepackage{setspace}  % Ettol a tablazatok, abrak, labjegyzetek maradnak 1-es sorkozzel!
\usepackage[hang]{caption}
\usepackage{hyperref}

\usepackage{graphicx}
\usepackage{epstopdf}

\usepackage{amsmath,mathtools,calc}
%piramisrajzoláshoz
\usepackage{tikz}
\usetikzlibrary{intersections,decorations.pathreplacing,shapes.misc,calc,positioning}

%RUDI vektoros betűtípus
\usepackage{lmodern}
%\usepackage{newtxtext}
%--------------------------------------------------------------------------------------
% Main variables
%--------------------------------------------------------------------------------------
\newcommand{\vikszerzo}{Ács Bence}
\newcommand{\vikcsapat}{CsapatX}
\newcommand{\vikcsapattagI}{Ács Bence}
\newcommand{\vikcsapattagII}{Jávorszky Zoltán}
\newcommand{\vikcsapattagIII}{Kálmán Bence}
\newcommand{\vikcsapattagIV}{Medvedev Mihály}
\newcommand{\vikcim}{Elektronikus jegyzőkönyv vezető rendszer tervezése UP módszerrel}
\newcommand{\viktanszek}{Szoftvertervezés - Házi Feladat}
\newcommand{\vikdoktipus}{Alkoholszondás vizsgálat jegyzőkönyve}
%\newcommand{\vikdepartmentr}{Bódis-Szomorú András}

%--------------------------------------------------------------------------------------
% Page layout setup
%--------------------------------------------------------------------------------------
% we need to redefine the pagestyle plain
% another possibility is to use the body of this command without \fancypagestyle
% and use \pagestyle{fancy} but in that case the special pages
% (like the ToC, the References, and the Chapter pages)remain in plane style

\pagestyle{plain}
%\setlength{\parindent}{0pt} % áttekinthetõbb, angol nyelvû dokumentumokban jellemzõ
%\setlength{\parskip}{8pt plus 3pt minus 3pt} % áttekinthetõbb, angol nyelvû dokumentumokban jellemzõ
\setlength{\parindent}{12pt} % magyar nyelvû dokumentumokban jellemzõ
\setlength{\parskip}{0pt}    % magyar nyelvû dokumentumokban jellemzõ

\marginsize{35mm}{25mm}{15mm}{15mm} % anysize package
\setcounter{secnumdepth}{0}
\sectionfont{\large\upshape\bfseries}
\setcounter{secnumdepth}{2}
\singlespacing
\frenchspacing

%--------------------------------------------------------------------------------------
%	Setup hyperref package
%--------------------------------------------------------------------------------------
\hypersetup{
    bookmarks=true,            % show bookmarks bar?
    unicode=false,             % non-Latin characters in Acrobat’s bookmarks
    pdftitle={\vikcim},        % title
    pdfauthor={\vikszerzo},    % author
    pdfsubject={\vikdoktipus}, % subject of the document
    pdfcreator={\vikszerzo},   % creator of the document
    pdfproducer={Producer},    % producer of the document
    pdfkeywords={keywords},    % list of keywords
    pdfnewwindow=true,         % links in new window
    colorlinks=true,           % false: boxed links; true: colored links
    linkcolor=black,           % color of internal links
    citecolor=black,           % color of links to bibliography
    filecolor=black,           % color of file links
    urlcolor=black             % color of external links
}

%--------------------------------------------------------------------------------------
% Set up listings
%--------------------------------------------------------------------------------------
%\lstset{language=C}
\lstset{
	basicstyle=\scriptsize\ttfamily, % print whole listing small
	keywordstyle=\color{black}\bfseries\underbar, % underlined bold black keywords
	identifierstyle=, 					% nothing happens
	commentstyle=\color{white}, % white comments
	stringstyle=\scriptsize\sffamily, 			% typewriter type for strings
	showstringspaces=false,     % no special string spaces
	aboveskip=3pt,
	belowskip=3pt,
	columns=fixed,
	backgroundcolor=\color{lightgray},
	inputencoding=utf8,
	extendedchars=true,
	literate={á}{{\'a}}1 {ã}{{\~a}}1 {é}{{\'e}}1,
} 		
\def\lstlistingname{lista}	

%--------------------------------------------------------------------------------------
%	Some new commands and declarations
%--------------------------------------------------------------------------------------
\newcommand{\code}[1]{{\upshape\ttfamily\scriptsize\indent #1}}

% define references
\newcommand{\figref}[1]{\ref{fig:#1}.}
\renewcommand{\eqref}[1]{(\ref{eq:#1})}
\newcommand{\listref}[1]{\ref{listing:#1}.}
\newcommand{\sectref}[1]{\ref{sect:#1}}
\newcommand{\tabref}[1]{\ref{tab:#1}.}

\DeclareMathOperator*{\argmax}{arg\,max}
%\DeclareMathOperator*[1]{\floor}{arg\,max}
\DeclareMathOperator{\sign}{sgn}
\DeclareMathOperator{\rot}{rot}
\definecolor{lightgray}{rgb}{0.95,0.95,0.95}

\author{\vikszerzo}
\title{\viktitle}
\includeonly{
	%guideline,%
	%project,%
	titlepage,%
%	declaration,%
%	abstract,%
%	introduction,%
%	chapter1,%
%	chapter2,%
%	chapter3,%
	chapter-glossary,%
	chapter-vision,%
	chapter-srs,%
	chapter-usecase,%
	chapter-domain,%
	chapter-usecase-realization,%
%	acknowledgement,%
%	appendices,%
}
%--------------------------------------------------------------------------------------
%	Setup captions
%--------------------------------------------------------------------------------------
\captionsetup[figure]{
%labelsep=none,
%font={footnotesize,it},
%justification=justified,
width=.75\textwidth,
aboveskip=10pt}

\renewcommand{\captionlabelfont}{\small\bf}
\renewcommand{\captionfont}{\footnotesize\it}

%
\newcommand{\fname}[1]{\mbox{\texttt{#1}}}

%--------------------------------------------------------------------------------------
% Table of contents and the main text
%--------------------------------------------------------------------------------------
\begin{document}
\singlespacing
%\include{guideline}
%\include{project}

\pagenumbering{arabic}
\onehalfspacing
%--------------------------------------------------------------------------------------
%	The title page
%--------------------------------------------------------------------------------------
\begin{titlepage}
\begin{center}
\includegraphics[width=60mm,keepaspectratio]{figures/BMElogo.png}\\
\vspace{0.3cm}
\textbf{Budapesti Mûszaki és Gazdaságtudományi Egyetem}\\
\textmd{Villamosmérnöki és Informatikai Kar}\\[1cm]
\textmd{\viktanszek}\\[4cm]

\vspace{0.4cm}
{\huge \bfseries \vikcim}\\[0.8cm]
\vspace{0.5cm}
\textsc{\Large \vikdoktipus}\\[4cm]

\begin{tabular}{cc}
 \makebox[14cm]{\emph{Készítette}}\\
 \makebox[7cm]{\vikcsapat}
\end{tabular}\\[1cm]


\begin{tabular}{cc}
	\multicolumn{2}{c}{\emph{Csapattagok}}\\
	\makebox[7cm]{\vikcsapattagI}&
	\makebox[7cm]{\vikcsapattagII}\\
	\makebox[7cm]{\vikcsapattagIII}&
	\makebox[7cm]{\vikcsapattagIV}
\end{tabular}

\vfill
{\large \today}
\end{center}
\end{titlepage}



\tableofcontents\vfill
%\include{chapter1}
%\include{chapter2}
%\include{chapter3}
%----------------------------------------------------------------------------
\chapter{Fogalomszótár}\label{sect:Glossary}
%----------------------------------------------------------------------------

%----------------------------------------------------------------------------
\chapter{Vízió}\label{sect:Vision}
%----------------------------------------------------------------------------
\section{Pozícionálás}
%----------------------------------------------------------------------------

%----------------------------------------------------------------------------
\section{Stakeholder leírások}
%----------------------------------------------------------------------------

%----------------------------------------------------------------------------
\section{Rendszer fő tulajdonságai}
%----------------------------------------------------------------------------

%----------------------------------------------------------------------------
\section{Rendszer funkcióinak összefoglalása}
%----------------------------------------------------------------------------

%----------------------------------------------------------------------------
\section{További igények és megszorítások}
%----------------------------------------------------------------------------

%----------------------------------------------------------------------------
\chapter{SRS - Kiegészítő követelmények leírása}\label{sect:SRS}
%----------------------------------------------------------------------------
\section{Bevezetés}
%----------------------------------------------------------------------------

%----------------------------------------------------------------------------
\section{Rendszer további funkciói}
%----------------------------------------------------------------------------

%----------------------------------------------------------------------------
\section{Használhatóság}
%----------------------------------------------------------------------------

%----------------------------------------------------------------------------
\section{Megbízhatóság}
%----------------------------------------------------------------------------

%----------------------------------------------------------------------------
\section{Teljesítmény}
%----------------------------------------------------------------------------

%----------------------------------------------------------------------------
\section{Technológiai megkötések}
%----------------------------------------------------------------------------

%----------------------------------------------------------------------------
\section{Támogatottság}
%----------------------------------------------------------------------------

%----------------------------------------------------------------------------
\section{Tervezési korlátozások}
%----------------------------------------------------------------------------

%----------------------------------------------------------------------------
\section{Interfészek}
%----------------------------------------------------------------------------
%----------------------------------------------------------------------------
\subsection{Felhasználói interfészek}
%----------------------------------------------------------------------------
%----------------------------------------------------------------------------
\subsection{Szoftver interfészek}
%----------------------------------------------------------------------------
%----------------------------------------------------------------------------
\subsection{Kommunikációs interfészek}
%----------------------------------------------------------------------------

%----------------------------------------------------------------------------
\section{Jogi vonatkozások}
%----------------------------------------------------------------------------

%----------------------------------------------------------------------------
\chapter{UseCaseModel}\label{sect:UseCaseModel}
%----------------------------------------------------------------------------
\section{Aktorok}
%----------------------------------------------------------------------------

%----------------------------------------------------------------------------
\section{UseCase-ek}
%----------------------------------------------------------------------------

%----------------------------------------------------------------------------
\section{A legfontosabb Use-Case-ek}
%----------------------------------------------------------------------------

%----------------------------------------------------------------------------
\section{A legfontosabb aktivitás diagramok}
%----------------------------------------------------------------------------

%----------------------------------------------------------------------------
\section{A legfontosabb szekvencia diagramok}
%----------------------------------------------------------------------------

%----------------------------------------------------------------------------
\chapter{DomainModel}\label{sect:DomainModel}
%----------------------------------------------------------------------------

%----------------------------------------------------------------------------
\chapter{UseCase Realization}\label{sect:UseCaseRealization}
%----------------------------------------------------------------------------

%----------------------------------------------------------------------------
\section{Szekvencia diagramok}
%----------------------------------------------------------------------------
%----------------------------------------------------------------------------
\section{Design Class diagramok}
%----------------------------------------------------------------------------


%\listoffigures\addcontentsline{toc}{chapter}{Ábrák jegyzéke}
%\listoftables\addcontentsline{toc}{chapter}{Táblázatok jegyzéke}

%TODO sorba számozza a hivatkozásokat + betűkicsinyítés?
%\bibliography{mybib}
%\addcontentsline{toc}{chapter}{Irodalomjegyzék}
%\bibliographystyle{plain}

%\include{appendices}

\label{page:last}
%Latex kisokos Ruditól
% ~ - elválasztás megakadályozása, pl 1.~ábra
\end{document}
